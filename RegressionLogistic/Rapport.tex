\documentclass[12pt]{article}
\usepackage[utf8]{inputenc}
\usepackage[T1]{fontenc}
\usepackage{amsmath}
\begin{document}
\part*{Regression logistique}
\section{Modèle}

\begin{equation}
Y=
\left\lbrace
\begin{array}{ccc}
0 & \text{:non(élève en réussite)}\\
1 & \text{: élève en réussite}\\
\end{array}\right.
\end{equation}

$X1$: Variable explicative binaire, signifiant l'appartenance ou non au GI
$X2$: Variable quantitative, signifiant le numéro du semestre
\end{document}
